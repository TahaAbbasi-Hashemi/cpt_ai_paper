The \gls{cpt} is a commonly used clinical test as it provides a acute controlled stressor to test the body's vascular and autonomic regulatory mechanisms. Part of the analysis of the brains autoregulaty capaabilities is to investigate the change in the \gls{mcav}. Thou, during \gls{cpt}, increases in \gls{mcav} may be blunted through pain-related hyperventilation and/or elevations in \gls{sna} \cite{elric}.
The \gls{cpt} is known to increase \gls{abp} which directly causes an increase in \gls{mcav} whereas a 1 mmHg decrease in \gls{petco2} can cause a 2.5\% decrease in \gls{mcav}. Given the pain-related hyperventation 


Why is waveform morphology important 

When we investigate the cardiovascular system we often find

an increased \gls{abp} is known to increase vascular stiffness, whereas an increased \gls{petco2} would decrease vascular stiffness. Vascular stiffness is a strong factor when it comes to changing the \gls{mcav}. Ultimately we found no statistically significant change for the \gls{mcav} with respect to time. This might indicate that while vascular stiffness increases it decreases by the same amount. Ultimately this fails to paint the full picture of the \gls{mcav}, while the absolute values didnt change the shape of the \gls{mcav} did change. The study of waveform morphology is still in its infancy where many of the respective features dont have concrete 