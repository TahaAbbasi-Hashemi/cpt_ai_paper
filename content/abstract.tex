The project aims to develop a predictive model for middle cerebral artery blood velocity (MCAv) responses during a 3-minute cold pressor test (CPT) using baseline blood velocity waveform morphological data. By analyzing morphological features extracted from MCAv waveforms, machine learning techniques can be used to predict diastolic, systolic, and mean MCAv values during a CPT protocol involving a single foot. The model predicted the three MCAv values as a 30-second average at the end of every minute. Data from a total of 33 participants (42\% male, age range 18-35 years) were included in this study. The three MCAv values are evaluated across four aggregation methods: no aggregation, aggregated input, aggregated output, aggregated input and output. Additionally, the study examined how different methods of splitting the dataset into testing and training sets affect the model’s error. The first method involved using 80\% of each participant’s waveforms for testing and the remaining 20\% for training. The second method split the dataset by population, with 80\% of participants were used for training and the remaining 20\% for testing. As time after intervention increased, the mean absolute error increased from 13.72 to 14.65 cm/s, likewise error increased going from diastolic to systolic, 10.494 to 18.269 cm/s respectively. Aggregating the input caused an increased error of 2.95 and 5.76 cm/s for participant and waveform split data respectively. Alternatively, aggregating both input and output had a lowered the overall error by 0.24 cm/s for participant split data whereas it increased it for waveform split data by 3.93 cm/s. Furthermore, aggregated input reduced the compute time from 6.65 minutes to 0.83 minutes. Aggregating both input and output for participant split data yields less biased predictions, minimizes computation time, and reduces error